% chapter01.tex

 %%%%%%%%%%%%%%%%%%%%%%%%%%%%%%%%%%%%%%%%%%%%%%%%%%%%%%%%%%%%%%%%%%%%%%%%%%%%%
 %                                                                           %
 %    PyMS documentation                                                     %
 %    Copyright (C) 2005-8 Vladimir Likic                                    %
 %                                                                           %
 %    The files in this directory provided under the Creative Commons        %
 %    Attribution-NonCommercial-NoDerivs 2.1 Australia license               %
 %    http://creativecommons.org/licenses/by-nc-nd/2.1/au/                   %
 %    See the file license.txt                                               %
 %                                                                           %
 %%%%%%%%%%%%%%%%%%%%%%%%%%%%%%%%%%%%%%%%%%%%%%%%%%%%%%%%%%%%%%%%%%%%%%%%%%%%%

\chapter{Introduction}

\section{Chromatography -- mass spectrometry}

PyMS is software for processing of chromatography--mass spectrometry
data. Mass spectrometry is an analytical technique widely used in chemistry
and biology for research, quality control studies, and forensic analysis.
Mass spectrometry analysis is based on detection ions separated by 
their different mass-to-charge (m/z) ratio. The ions can be generated
by a variety of methods (thermal ionization, chemical ionization,
electron impact ionization, etc), and can be ionized atoms or molecular
fragments. The separation by m/z ratio could also be achieved in a
variety of ways (static or dynamic electric or magnetic fields). The
combinations of ionization methods and ion separation methods leads
to a great variety of mass spectrometers. As a consequence today it
is possible to purchase many different mass spectrometer instruments
which can be used in a variety of experimental setups. For more
information see \cite{gross04}.

{\em Gas chromatography}.
Chromatography separation relies on a column packed with a stationary
phase (immobile) over which a mobile phase is pushed. The mobile phase
is either gas (gas chromatography, GC) or liquid (liquid chromatography,
LC). The mobile phase carries the complex mixture to be separated. The
two phases are chosen so that components of the mixture have different
solubilities in each phase. Each component of the mixture is in
equilibrium between the stationary and the mobile phase, determined
by its solubility in the two phases. As a result components of the
mixture (different molecules) travel with different speed through
the column, resulting in the separation of the molecular mixture.

{\em Mass spectrometry}.
Mass spectrometry analyzer is located at the outlet of the chromatography
column. As the mobile phase leaves the column (possibly with some
component of the sample mixture) the mass spectrometer records a full
mass spectrum (m/z vs intensity). Recording one such spectrum is called
a "scan", and mass spectrometer operates in repetitive scanning mode.
The separation occurs on the time scale of minutes, and a certain number
of scans is performed per unit time to record mass spectra continuously.

{\em Hyphenated mass spectrometry data}.
Joining chromatography separation with mass spectrometer results in
hyphenated mass spectrometry, GC-MS or LC-MS, depending on the nature
of the chromatoraphic separation.
The time between the sample injection and the detection of an analyte
on the detector at the end of the column is called the retention time.
Retention time is unique for the analyte, i.e. each compound in a
mixture. The data sets resulting from GC-MS and LC-MS are complex.
The raw data could be viewed as a two dimensional matrix, with the
retention time along one dimension. For each time point a mass spectrum
appears in the second dimension. Often the data is viewed as the
projection along the time axis: for each time point all m/z values
are summed to give one-dimensional spectrum (total ion chromatogram
or TIC).  Gas chromatography coupled to mass spectrometry (GC-MS) is
suitable for profiling of volatile, thermally stable metabolites (or
metabolites made such by chemical derivatization), while liquid
chromatography coupled to mass spectrometry (LC-MS) is suitable for
profiling of predominantly polar metabolites \cite{halket05}.

The principles of GC-MS and LC-MS data processing are well established.
A typical data processing pipeline may involve noise attenuation,
baseline correction, peak detection, and peak quantitaion (integration).
Overlapping signals may be discerned by processing ion chromatograms
for individual masses (deconvolution), or two-dimensional data
processing methods may be applied directly on the spectral matrix.
In GC-MS mass spectra for individual time points are often matched
against large libraries of mass spectra for compound identification
purposes.

\section{Metabolite profiling (metabolomics)}

Metabolite profiling is an emerging field of functional genomics,
highly complementary to transcript profiling (transcriptomics) and
protein profiling (proteomics), with significant promises for
applications in biomarker discovery, toxicology, nutrition, basic
metabolic research and integrated systems biology \cite{fiehn00,allen03}.
Metabolomics refers to an all-inclusive profiling of low-molecular
weight metabolites with an implicit aim to interpret the results
in the context of the organism's genome and its global metabolic
network. Hyphenated mass spectrometry and nuclear magnetic resonance
(NMR) are the principal analytical techniques in metabolomics.
Chromatography separation coupled to mass spectrometry detection
is a robust analytical approach used to quantitate hundreds of
compounds in metabolomic studies.

Non-targeted metabolite profiling has brought about new challenges
in the development of data processing methods able to support
large scale, high-throughput experiments. Data processing for
high-throughput metabolomic experiments is still more a defining
goal than reality, and may require utilization of emerging computing
platforms such as distributed and Grid computing to speed up the
processing. A related challenge in the domain of bioinformatics
is the effective application of statistical, machine learning,
and data mining methods required to extract useful information
from data.

\section{About PyMS}

PyMS is a modular software package for processing of
chromatography--mass spectrometry data with emphasis on scripting
capabilities. The idea behind PyMS is to decouple processing
methods form visualization and the concept of interactive
processing altogether. The purpose of this is to provide a set
of components for rapid development and testing of new processing
methods and algorithms, as well as automated data processing.

PyMS is released as open source, under the GNU Public License
version 2.  

